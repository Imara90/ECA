% A New Architecture for Trustworthy Autonomic Systems

Validation alone does not always guarantee trustworthiness as each individual decision could be correct, but the overall system might not be consistent or dependable. These aspects should be integrated at architectural level and not be seen as add ons. Autonomic computing is mostly based on the architecture's basic MAPE (monitor, analyze, plan, execute) control loop. Another inspiration for autonomic systems is Intelligent Machine Design (IMD), based on the human autonomic nervous system.

In large systems with a wide behavioral space it is highly complex to determine whether all autonomic decision were in overall interest of the system. There is a vital need to dynamically validate the run-time decisions of the autonomic manager. The higher goal is not to just reach self-management, but to achieve consistency and reliability of results through self-management.

Current research has looked into a fifth state of the self* properties: self-regulation. It tests itself integral in the architectural, however it assumes that the other states perform optimally and they do not ensure trustworthiness. Another believe is that trustworthiness is achieved when keeping an account of its behavior. This requires the user to intervene if necessary. A dead-zone can be introduced to prevent unnecessary inefficient and ineffective control brevity when the system is sufficiently close to its target value.

The proposed trustworthy architecture exists out of  Autonomic Controller, an Validation Check, a Dependability Check and the managed sub-system. The AC doesn't matter about the content of the unit, but only provides an interface to the designers to express rules that govern the goal. The VC is a higher level mechanism that keeps track of the goal. It is important to also consider the possibility of overall inconsistency in the behavior of the system (the AM could erratically be changing its mind, causing oscillation). The DC only allows the AM to change its mind when its necessary and safe to do so.  Dead-zone logic is implemented to account for this, as well as prediction and learning. A system, no matter the context of deployment, is truly trustworthy when its actions are continuously validated to satisfy set requirements.
