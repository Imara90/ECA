%Survey of Frameworks, Architectures and Techniques in Autonomic Computing

Self-configuration is the ability of the system to perform configurations according to the pre-defined high level policies and seamlessly adapt to change caused by automatic configurations. Self-optimization is the ability of the system to continuously monitor and control resources to improve performance and efficiency. Self-healing is the ability of the system to automatically detect, diagnose and repair faults. Self-protection is the ability of the system to pro-actively identify and protect itself from malicious attacks or cascading failures that are not corrected by self-healing measures.

This paper presents a wide variety of techniques in autonomic computing. Hot-swapping is a technique to inject monitoring and diagnostic code into live code using inter-positioning and replacement. Data clustering is an unsupervised learning algorithm, used to identify configuration classes and determines the degree of similarity between clusters using convex average metric.

