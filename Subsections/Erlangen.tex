% The Erlangen Slot Machine: A Dynamically Reconfigurable FPGA-based Computer

[There has to be some images added or else it doesn't come across, or just less info] 

The Erlangen Slot Machine from \cite{erlangen} tackles the four major limitations of the Virtex-II FPGA produced by Xilinx. First, the I/O dilemma caused by fixed pins spread around the device is solved by connecting all bottom pins from the FPGA to an interface controller realizing a crossbar. It connects FPGA pins to peripherals automatically based on the slot position of a placed module. This I/O rerouting principle is done without reconfiguration of the crossbar FPGA.

Second, the memory dilemma has been solved. In normal Virtex-II FPGAs, a module can only occupy the memory inside its physical slot boundary. Storing data in off-chip memories is therefore the only solution. In the ESM, six SRAM banks are connected to the FPGA. Since these banks are placed at the opposite side as the crossbar, a module will connect to peripherals from one side, while the other side will be used for temporally storing computational data. In order to use a SRAM bank (called a slot), the module must have at least a width of three micro-slots, in which the total device is divided. This organization simplifies relocation, enabling a partially reconfigurable computing system. Also, equal resources will be available for each module.

Finally, the inter-module communication dilemma is dealt with. Dynamically routing signal lines on the hardware is a very difficult task. The ESM uses a combination of bus-macros, shared memory, RMB (Reconfigurable Multiple Bus) and a crossbar to take away the limiting factor for the wide use of partial dynamic reconfiguration.

In order to initialize executable application modules and their run-time supervision the ESM requires an operating system. This is being implemented by means of a Reconfiguration Manager that uses a parallel port interface to download and store bitstreams into the flash memory. A pipelined data flow architecture has been used, replacing the finite state machine by a MicroBlaze microcontroller and employing a data crossbar between plug-ins. Providing  a new architecture to avoid the current physical problems of reconfigurable FPGAs, a new inter-module communication concept, as well as an intelligent module reconfiguration management has made the ESM an alternative for the Xilinx FPGAs.
