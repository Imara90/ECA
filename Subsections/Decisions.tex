\subsection{Decision making techniques}
\label{sec:decisions}

For a self-adaptive system the intelligence in using the monitoring data is equally important to collecting the data. Results from the monitoring framework are fed to a decision making application. Making decisions for autonomic systems is the second action in the closed-control loops such as the \emph{observe-decide-act (ODA) loop}. Decision making all depends on the input and constraints given to the system. It can either concern a best-performance decision making or a sufficiently good performance that respects the user's performance goals \cite{evolvable}. Closed-control loops are implemented by the operating system reciding on the reconfigurable hardware. By using dynamic self-adaptive libraries, the operating system can choose at runtime which is the best suiting implementation for the required functionality among the available \cite{evolvable}.

A desicioning framework can be divided into 2 categories: analytical models and empirical models \cite{evolvable}. The analytical models are good when working in the same environment as they are manually generated and are very precise because of this prior knowledge. However when considering evolvable systems in dynamic environments, empirical models are favourable as they exploit information collected at run-time, but are in turn less accurate.
Current trends in supporting decision making are using heuristic policies, machine learning, control theory and competitive algorithms. Heuristics is the application of experience-derived knowledge to a problem. Heuristic software can be easily developed, applied and reused when working within a known environment including static condition. However within the context of a dynamic live application with changing environments, it is very easy to miss unforeseen problems using software that looks for known problems.
% weet niet of ik dit als source moet gebruiken: http://searchsoftwarequality.techtarget.com/definition/heuristics

