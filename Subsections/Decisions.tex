\subsection{Decision making techniques}
\label{sec:decisions}
%
Since it is impossible to pre-configure all possible scenarios in a dynamic system, learning and decision engines are introduced in self-adaptive systems. Results from the monitoring framework are fed to the engines. Decision making all depends on the input, the predefined constraints and the expected \emph{Quality-of-Service (QoS)} \cite{evolvable}. Closed-control loops are implemented by the operating system reciding on the reconfigurable hardware. By using dynamic self-adaptive libraries, the operating system can choose at runtime which is the best suiting implementation for the required functionality among the available \cite{evolvable}.

A desicioning framework can be divided into 2 categories: analytical models and empirical models \cite{evolvable}. The analytical models are good when working in the same environment as they are manually generated and are very precise because of this prior knowledge. However when considering evolvable systems in dynamic environments, empirical models are favourable as they exploit information collected at run-time, but are in turn less accurate.
Current trends in supporting decision making are using heuristic policies, machine learning, control theory and competitive algorithms. Heuristics is the application of experience-derived knowledge to a problem. Heuristic software can be easily developed, applied and reused when working within a known environment including static condition. However within the context of a dynamic live application with changing environments, it is very easy to miss unforeseen problems using software that looks for known problems.
% weet niet of ik dit als source moet gebruiken: http://searchsoftwarequality.techtarget.com/definition/heuristics

%The operating system is in charge of choosing among possible implementations at run-time. Criteria for these decisions can be expected performance, available resources, input data type and size, the functionalities that are already implemented and the availability of hardware components.