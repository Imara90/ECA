% --- SW Proposal --------------------------------------------
\subsection{Proposed Software}
\label{sec:prosoftware}
The evolvable framework co-designed with the proposed hardware has to include a monitoring technique, a decision making engine and an adaptation framwork. The Evolutionary Framework proposed in \cite{PDR} is based on an evolutionary algorithm that uses static input data files as reference for its evolution. This section proposed an adaptable fully autonomous system on top of our heterogeneous architecture. 

The \emph{HeartBeats application} is proposed as the monitoring technique as it enables the system to predefine performance goals and delivers real-time information to the decision engine. This will introduce a moderate overhead by initializing the data structures which will relatively decrease when dealing with larger complex systems. This is acceptable as Section \ref{sec:codesign} displayed that the use of evolvable system only becomes valid when working with these complex systems. This run-time information will support an agile self-aware system. 

The monitoring system will feed the performance information to the decision engine. Inspired by the PDR methods a partial dynamic model is proposed. A combination of an \emph{analytical} and \emph{empirical model} will enhance the initial performance of the system as it has priorly developed knowledge in the analytical model. The empirical model can take over when performance goes out-of -bounds to handle unexpected behavior. 

PDR will reduce the amount of overhead introduced by reconfiguration during self-adapting. However a framework to adapt, or switch between implementations of functionalities still has to be provided. The \emph{Implementation Switch Service} inspired by the \emph{hot-swap mechanism} as proposed in \cite{selfaware} solves the state quiescence and state translation and thus completes the proposed evolvable framework.
