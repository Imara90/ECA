% Software based approaches
The \emph{Heartbeats application} is a monitoring application which makes it possible to assert performance goals as heart-rate windows delimited by a minimum and maximum performance, or \emph{heart-rate} \cite{evolvable}. The Heartbeats API is made of small straightforward functions and allows declaring performance goals. Software components first have to register, specifying parameters such as minimum and maxmim heart-reate, size of the windows of observation and history buffers \cite{selfaware}. The application then updates the progress of the execution calling the function that signifies a heartbeat \cite{evolvable}.

%[SMARTLOCKS PAPER] presents a combination of the Heartbeats application in cooperating with other frameworks. Smartlocks is a spinlock library that will adapt its internal implementations based on defined goals during run-time using heurstics as discussed in subsection \ref{sec:decisions} and machine learning. 

\cite{reconfigurable} dscusses using adaptive programming in situations where input changes lead to relatively small output changes. They present a hardwar/software codesign paradigm that develops a new performance model and associated evaluation metrics to differentiate between various levels of performance across different portions of software modules. Incoorporating this into a codesign environment increases the flexibility of the system. 
%distributed self trained algorithms, machine learning, heartbeats application
%adaptive programming