\subsection{Monitoring techniques}
\label{sec:selfawareness}
Most current architecture include at least basic hardware to assist in system monitoring often in the form of counter registers. These techniques suffer often from shortcomings such as not enough counters, a sampling delay and a lack of address profiling \cite{reconfigurable}, which make them unreliable in a dynamic enivironment that should be able to deal with unexpected inputs. These techniques only enable collecting statistics using sampling and lack the possibility to react and thus unfit as a monitoring element in the observe-decide-act loop. Current systems tried to address these shortcomings by introducing microarchitectural event data that can be delivered to the operating system through an exception which would imply the use of frequent interrupts \cite{reconfigurable}. Interrupts would however introduce an overhead for they need state saves and translations. 

The heartbeats application as discussed in Section \ref{sec:software} is a technique used for monitoring. Using a framework like this is particularly convenient because it allows to automatically update all information concerning the global heart-rate while concurrently updating its external observers, such as decision making engines. The heartbeats application enables the operating system to directly react and receive information compared to the performance goals. Although \cite{selfaware} presents an average overhead of 3,52\% due to system calls in order to initialize its data structures and update the global heart rate. This overhead is moderate concidering it is the fundamental part for an autonomic system and the percentage of overhead caused by the initialization of the data structures wil decrease when working with larger complex systems. 