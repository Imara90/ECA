\subsection{Self-monitoring techniques}
\label{sec:selfawareness}
A system is self-aware when it is constantly aware of its internal state. This is the first step towards an autonomic system. Autonomic systems are build up of autonomic elements that include sensors in order to monitor its behavior. A system is self-monitoring when it has knowledge of its available resources, its components, the desired performance characteristics, the current status and the status of the inter-connections with other systems \cite{autonomic}. The knowledge obtained from monitoring is utilized in \emph{closed-control loops} such as the \emph{observe-decide-act (ODA) loop}. Given certain performance goals these loops \emph{observe} the current state, decide upon an action and perform these actions as will be discussed in section \ref{sec:decisions} and \ref{sec:selfadapting}.

The \emph{Heartbeats application} is a monitoring application which makes it possible to assert performance goals as heart-rate windows delimited by a minimum and maximum performance, or \emph{heart-rate} \cite{evolvable}. The Heartbeats API is made of small straightforward functions and allows declaring performance goals. Software components first have to register, specifying parameters such as minimum and maxmim heart-reate, size of the windows of observation and history buffers \cite{selfaware}. The application then updates the progress of the execution calling the function that signifies a heartbeat \cite{evolvable}. A framework like this is particularly convenient because it allows to automatically update all information about the global heart-rate which is then made available to external observers, such as decision making applcations. However it introduces an average overhead of 3,52\% due to system calls in order to initialize its data structures and updating the global heart rate \cite{selfaware}.