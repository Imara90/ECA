\subsection{Self-awareness techniques}
\label{sec:selfawareness}
Self-aware adaptive computing systems are capable of adapting their behavior and resources thousands of times based on changing environmental conditions and demands \cite{selfaware}. 
This allows them to automatically accomplish their goals in the best way possible. Autonomic systems are build up of autonomic elements. These elements must include sensors and effectors \cite{autonomic} in order to monitor its behavior and in turn become self-aware. \emph{Closed-control loops} are often incoorporated to monitor the behavior considering the given performance goals. In subsection \ref{sec:decisions} closed-control loops are further discussed. 

The \emph{Heartbeats application} is a monitoring application which makes it possible to assert performance goals as heart-rate windows delimited by a minimum and maximum performance, or \emph{heart-rate} \cite{evolvable}. The Heartbeats API is made of small straightforward functions and allows declaring performance goals. Software components first have to register, specifying parameters such as minimum and maxmim heart-reate, size of the windows of observation and history buffers \cite{selfaware}. The application then updates the progress of the execution calling the function that signifies a heartbeat \cite{evolvable}. A framework like this is particularly convenient because it allows to automatically update all information about the global heart-rate which is then made available to external observers, such as decision making applcations. However it introduces an average overhead of 3,52\% due to system calls in order to initialize its data structures and updating the global heart rate \cite{selfaware}.