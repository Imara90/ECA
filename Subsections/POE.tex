% A Phylogenetic, Ontogenetic, and Epigenetic View of Bio-Inspired Hardware Systems

In artikel \cite{poe} a new kind of engineering is being introduced. Inspired by life on Earth and the natural processes of living organisms, \emph{bio-inspired} hardware systems have evolved. The paper introduces the POE model, that classifies bio-inspired systems according to three axes:
\begin{itemize}
	\item Phylogeny, which concerns the evolution of a species over time, aiming for optimization of the genome
	\item Ontogeny, which is about a second biological organization within multicellular organisms, being celluar reproduction
	\item Epigenesis, concerning the learning systems of organisms such as the nervous system, the immune system and the endocrine system.
\end{itemize}

Along the axis of phylogeny, evolvable hardware can be found. Artificial evolution and large-scale programmable circuits are two underlying themes. Evolutionary algorithms are common nowadays, as they strive for optimization and automatic programming. Programmable circuits are integrated circuits that are to configured by the user. Back in the days, the PAL (programmable array logic) was the most popular PLD (programmable logic devices). Nowadays, field programmable gate arrays (FPGAs) have become widely used, providing high flexibility and the possibility of being reconfigurable.

Ontogenetic systems have the ability to self-repair, as the main goal of ontogeny is growth or construction. Whereas phylogeny is about reproduction of the system by use of crossover and mutation, ontogeny is about replication by creating daughter cells that are exact duplicates of the mother cell.

Epigenetic systems are more software-based, as they contain error detection and immune systems for computers. It becomes interesting when multiple axes are being combined, as stated at the end of \cite{poe}. By dreaming about POE hardware systems that are endowed with evolutionary, reproductive, regenerative learning capabilities this article is often \emph{geraadpleegd} in research considering evolvable systems.