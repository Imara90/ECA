% Self-aware adaptation in FPGA-based systems
 
The results of \cite{selfaware} presents a system built on top of a set of enabling technology that proves the effectiveness of using self-aware adaptive computing systems.
Self-aware adaptive computing focuses on creating a balance of resources to improve performance, utilization, reliability and programmability. 
A programmer will ideally only have to provide the system its goal, rather than a description of tasks, provided with some constraints.

In a system like this, the hardware, the application and the operating system have to be seen as unique entities to autonomously adapt itself. 
The underlying architecture has cognitive hardware mechanisms in its core to observe and affect the execution. 
The self-aware adaptive computing system also implements learning and decision making engines  to determine appropriate actions. 
A key challenge is to identify what parts of a computer need to be adapted.
