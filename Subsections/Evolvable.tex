% Evolvable systems on reconfigurable architecture via self-aware adaptive applications. ++

They propose an evolvable system that runs self-adaptive applications on top of a heterogeneous system consisting of a general purpose processor and a reconfigurable device. The operating system running on top of the heterogeneous system is responsible for providing self-adaptive capabilities. Specifications are running a customized version of GNU/Linux and a set of self-adaptive applications on top of a heterogeneous system featuring a multi-core processor, an Intel Core i7 and a reconfigurable device: a Xilinx Vertex 5 FPGA based board.

There is the increasing importance of non-functional requirements such as power consumption and reliability as well as many potentials that lie at the border of functional such as efficiency and accuracy. Meeting such constraints is getting more and more difficult , mainly because of the exponential increase of interactions among systems and the environments in which the systems are required to work. They also enable self-adaptiveness through a monitoring framework, which tackles the primitive layer of self-adaptiveness that it self-awareness and through an adapting framework which deals with self-configuration and self-optimization, the major layer of self-adaptiveness.

The monitoring process is central to self-adaptive systems. The application Heartbeats is particularly effective for implementing these applications. Using machine learning, one can enhance synchronization techniques. K32 is a modern object-oriented operating system that responds to changing and challenging environments adapting its operations. It supports on line reconfiguration of functionalities by interposition, hot-swap and dynamic update.

Evolvable systems exploiting self-adaptive techniques are self-configuring and self-optimizing systems capable of changing their operations to meet the given performance goals by modifying either the underlying heterogeneous architecture, the operating system and the self-adaptive applications.

Design choices are important when developing the observe-decide-act loop. Their vision is that the underlying hardware architecture is made up of  static area and a dynamic area. The reconfigurable device can be configured to implement different functionality through dynamic reconfiguration support provided by the operating system. This is provided through standard libraries and the OS implements parts of the loop. The OS is therefor capable of choosing at run-time the best implementation for the required functionality among the available.

Heartbeat makes it possible to assert performance goals as heart-rate windows each of each is delimited by a minimum and maximum heart-rate. It updates the progress of the execution calling the function that signifies a heartbeat. It monitors the progress of the execution through either a windows heart rate and a global heart-rate.

A decisioning framework can designed using analytical models and empiric models. They choose to implement empiric models because of the generality and for the reduced effort they place on the designer. The hot-swap mechanism then provides the ability of switching among different implementations of the same functionality in a transparent function while executing.  The switchable units were identified as the libraries that export an implementation of a certain functionality, a self-adaptive library or Dynamic-link library (DLL).

They created 2 libraries, one for the multi-core processor and one for the FPGA. The former is programmed using C and where necessary assembly to further refine the performance.
