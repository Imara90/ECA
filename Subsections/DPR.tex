% A Fast Reconfigurable 2D HW Core Archtiecture on FPGAs for Evolvable Self-Adaptive Systms

Another FPGA-based architecture is proposed in \cite{dpr}. A highly regular and modular architecture is being integrated with a widely known 2D systolic processing architecture with an optimized DPR control engine. The engine allows the implementation of adaptive (evolvable) prosessing-hardware with native reconfiguration support. In the design, the processing elements (PEs) of the reconfigurable core are structured as a 2D systolic array, known for its high performance and restrained use of resources (for collection of processed data only the lowest and rightmost PE has to be considered).

As for the reconfiguration engine, three enhancements are included in order to achieve fast reconfiguration. First, only the body of the bitstream is stored. The header and tail info are eliminated and are added at run time. This leads to reduction of the bitstream size (and thus, data transference time from the external memory) and raster relocation possibilities. Second, internal memories have been included, avoiding  the same element to be reallocated at different positions int he architectue. This greatly reduces the reconfiguation overhead, which is a limitation when using VRCs. Another technique that has been applied is overclocking the Virtex-4 Internal Configuration Access Port (ICAP) to 2,5 times the maximum frequency report by the manufacturer. This was no problem and also reduced the reconfiguration overhead.
