% A Fast Reconfigurable 2D HW Core Archtiecture on FPGAs for Evolvable Self-Adaptive Systms

Another Virtex-4 based architecture introduced in \cite{PDR} applies the 2D reconfiguration core as well. Rather than direct bitstream manipulation, a \emph{Partial Dyanmically Reconfiguration (PDRR)} control engine has been integrated. As a result, the processing elements (PEs) of the reconfigurable core are structured as a 2D systolic array, known for its high performance and restrained use of resources. The reconfiguration engine has been optimized by applying three enhancements that will be discussed in \ref{sec:PDR}. 

\cite{PDR} also proposes an adaptation framework to match this architecture. An \emph{Evolutionary Algorithm (EA)} runs on the MicroBlaze processor with an Evaluation Module (EM) and a tightly coupled on-chip RAM memory.  Inspired by Cartesian Genetic Programming a simple \emph{(1 +$\lambda$)} Evalution with 1 parent and $\lambda$ offspring has been implemented. From a random initial population, their selection algorithm chooses the fittest individual with elitism enabled as a parent for the next generation. The evaluation can be divided into three phases:

\begin{itemize}
	\item \emph{Reconfiguration} of the Core using the candidate selected by the EA
	\item \emph{Execution} of the processing task confgured in the evolvable core
	\item {Fitness} computation of the solution
\end{itemize}

The last two phases can run in parallel, however an efficient fitness function is not declared. The EA also needs datasets of \emph{InputData} and the \emph{IdealResult}. These are provided using an external FLASH memory and are thus statically created prior to running the application which is a downside of the system. An evolvable system needs to be able to cope with unexpected input data in a live environment.