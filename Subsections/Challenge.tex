%Self-managed Systems: An Architectural Challenge

A self-managed software architecture is one in which component automatically configure their interaction in a way that is compatible with an overall architectural specification to achieve the goals of the system. Dynamic change which occurs while the system is operational, is far more demanding and requires the system to evolve dynamically and that adaption occurs at run-time. There is a community called SEAMS (Software engineering for adaptive and self-managing systems).

They focus on the use of ADLs for software design and implementation from components, including limited language support for dynamic change, a general model for dynamic change and evolution, associated analysis techniques and initial steps towards self-management.

The three-layer Gat model is presented. The bottom control layer consists of sensors, actuators and control loops and includes facilities to report the status to the middle layer. The change management layer, or middle layer reacts to state changes accordingly and can introduce new components. The upper goal layer consists of time consuming computations such as planning. They propose a component design that implements the set of services that it provides and the set that it may need.

