% -- Introduction ---------------------------------------

\section{Introduction}

For the lab a computational intensive kernel has to be extracted from a \mcode{x264} software application, a free software library for encoding video stream into H.236/MPEG-4 AVC format. It is able to use Periodic Intra Refresh instead of keyframes (which is used by h.264), refreshing the image of the video by moving a column of intra blocks from one side of the screen to the other. This hides the refreshing effect from the user while the frame loads. The \mcode{x264} application is executed on a FPGA running a MicroBlaze host processor. By running the particular kernel on the $\rho$-VEX co-processor, the execution time can be decreased by making use of parallel processing. 
\emph{Er is heel weinig informatie te vinden over de precieze werking van x264, aangezien er een patent opzit.}

\subsection{Detecting the Computationally Most Intensive Kernel}
The first lab is meant to learn about the execution specifics of the application. A few input files are given to show the compile and run commands of the x264 application, by inserting a \mcode{.y4m} stream and creating a short \mcode{.mkv} movie. By adding the \mcode{gprof} flag to the compile command, a list is created of all functions ordered by their share of the total execution time (in percentage). Also the number of function calls is shown, as well as the total execution time for each input file. The input files exist of five \mcode{.y4m} files, varying in resolution (either 64x32 or 640x320 pixels) and amount of frames (either 1, 3, 8, 32 or 128).

Profiling the x264 execution for the \mcode{.y4m} files that are provided by the lab leads to the ranking show in table \ref{tab:chart}.

% -- GPROF uitslagen ---------------------------------------
\begin{table}[htb]%
\centering
\small
	\begin{tabular}{lclll}
		\centering
		\bf{Input file} & \bf{Time (sec)} 	& \bf{Share ($\%$)} & \bf{Calls}	& \bf{Kernel name} \\ \cline{1-5}
		\multirow{3}{*}{eledream\_32x18\_1.y4m}	&				& 100.00	& 71		&	x264\_analyse\_init\_costs\\ 
																						&	0.02	& 0.00 		& 1646	&	x264\_free\\ 
																						&				& 0.00		& 784		&	x264\_cabac\_encode\_desicion\_c\\ \cline{1-5}
		\multirow{3}{*}{eledream\_64x32\_3.y4m} & 			& 66.67		& 71		& x264\_analyse\_init\_costs\\
																						&	0.03	& 33.33 	& 1971	&	x264\_pixel\_satd\_4x4\\ 
																						&				& 0.00		& 4830	&	x264\_pixel\_satd\_8x4\\ \cline{1-5}
		\multirow{3}{*}{eledream\_640x320\_8.y4m}	& 			& 14.29	& 1599044		&	x264\_pixel\_satd\_8x4\\ 
																							&	1.61	& 11.80 	& 570708	&	x264\_get\_ref\\ 
																							&				& 4.97		& 38770		&	x264\_pixel\_satd\_x4\_16x16\\ \cline{1-5}
		\multirow{3}{*}{eledream\_640x320\_32.y4m}& 			& 20.61		& 7228633		&	get\_ref\\ 
																							& 8.54	& 13.23 	& 15386501	&	x264\_pixel\_satd\_8x4\\ 
																							&				& 4.57		& 1009690		&	x264\_pixel\_satd\_x4\_8x8\\ \cline{1-5}
		\multirow{3}{*}{eledream\_640x320\_128.y4m}&			& 17.48		& 21956292	&	get\_ref\\
																							&	 29.86& 14.17 	& 49862831	&	x264\_pixel\_satd\_8x4\\
																							&				& 6.56		& 1023315		&	x264\_pixel\_satd\_x4\_16x16\\ \cline{1-5}
	\end{tabular}	
\caption{Chart with computationally most intensive kernels for each input stream.}
\label{tab:chart}
\end{table}

\subsection{The \mcode{pixel_satd_8x4} kernel}
Given these statistics, we decide to extract the \mcode{x264\_pixel\_satd\_8x4 kernel}, since its share in execution time increases as the files become larger. This kernel evaluates the Sum of Transformed Differences (SATD) between a 8x4 pixel block from the input stream and reference blocks. The SATD is a metric used for video compression where the differences between the pixels are taken and put into a frequency transform, usually a Hadamard transform. The two pixel blocks are being compared and the Hadamard keeps track of which bytes have a match and which does not by means of a Boolean 4x4 matrix as can be seen in figure HADAMARD.

*** plaatje Hadamard ***

\subsection{Executing an Application on the Development Board and $\rho$-VEX}

When executing \mcode{./configure}, a file is created for configuration of the application. However, the resulting \mcode{config.mak} file is made for applications running on the guest (Ubuntu). In order to configure for MicroBlaze, the \mcode{config.mak} file has to altered. All references to \mcode{m32} have to be removed and the \mcode{--DWORDS\_BIGENDIAN} flag has to be added to the \mcode{CFLAGS} variable. This has to be done everytime when configurating the application for MicroBlaze.

After doing this, the application now can be 'made' for MicroBlaze by first moving to the Scratchbox 2 environment for MicroBlaze and then execute the \mcode{make} command. Fig. \ref{fig:lelijk} shows a block diagram of both the platforms.

% -- Plaatje Practicum Evnvironments ------------------------------
\begin{figure}[htb]%
\centering
\includegraphics[width=300px]{Pictures/Platform_paint}%
\caption{Block diagram of the Virtual Machine and the ERA platform}%
\label{fig:lelijk}%
\end{figure}

In order to run the application on the MicroBlaze host processor, the \mcode{x264} executable must be compressed along with an input stream in a \mcode{tar.gz} file. This file can be put, via the FPGA host machine, on one of the three FPGAs using the \mcode{scp}, \mcode{ftp} and \mcode{put} command. For the programmer to connect to the development board, one must use the \mcode{ssh} and \mcode{telnet} command. See also fig \ref{fig:hoppen}.

% -- Plaatje Ubuntu-commando's ------------------------------
\begin{figure}[htb]%
\centering
\includegraphics[width=300px]{Pictures/hop}%
\caption{Ubuntu commands for putting files on and navigating to the FPGAs}%
\label{fig:hoppen}%
\end{figure}

For the first lab, the \mcode{satd\_8x4} kernel has to be:
\begin{itemize}
	\item extracted into a separate file
	\item supplemented to a proper .c file which can be compiled and run using the $\rho$-VEX
	\item included in the makefile that creates a bytecode and bytedata file of the kernel
	\item debugged
\end{itemize}

Looking back, the debugging part has been chasing us all the way through the lab. Not only did we suffer from bugs in our source code, the $\rho$-VEX itself did also have shortcomings that had to be circumvented by downloading several fixes, competing for FPGA availability and break downs of the entire host machine due to unsufficient capacity for the amount of students.

\subsection{Memory layout of the MicroBlaze}

When writing pixels to the $\rho$-VEX for SATD calculation, these pixels need to be picked from the MicroBlaze memory. However, these pixels are not written as one block of contiguous bytes. Rows of four bytes (MicroBlaze has 32-bit registers) are interrupted by a fixed amount of memory called a stride. This value is often given in bytes to be skipped in order to continue reading the particular data file. Figure \ref{fig:stride} shows the concept of strides for two pixels from the MicroBlaze memory, which are splitted into two rows of 4 bytes since a pixels consists of eight bytes. When writing data to the $\rho$-VEX memory we use a stride of four bytes, resulting in a configuration as shown in \ref{fig:testpixels}

% -- Plaatje Stride ------------------------------

\begin{figure}[htb]
	\centering
	\begin{subfigure}{0.3\textwidth}
		\centering
		\includegraphics[width=150px]{Pictures/stride}
		\caption{Large unknown stride}
		\label{fig:stride}
	\end{subfigure}
	\quad
	\begin{subfigure}{0.3\textwidth}
		\centering
		\includegraphics[width=150px]{Pictures/pixels_dmem}
		\caption{Stride of four bytes}
		\label{fig:testpixels}
	\end{subfigure}
\caption{MicroBlaze memory using strides to split up data files}%
\label{}%
\end{figure}
