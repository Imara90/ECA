% Discussion of the papers

\section{Discussion of the Different Papers}
\label{sec:discussion}
Considerable research has already been done in order to efficiently accelerate hardware while still maintaining virtually unlimited adaptability. Software techniques in autonomic computing systems such as hot-swapping and data clustering are discussed in \cite{survey}. \cite{selfaware} proposes a self-aware system able to adapt the behavior of FPGA-based systems.
Other approaches start with a FPGA-based architecture with a reconfigurable core \cite{PDR}, added programming schemes and new cell structures as seen in \cite{virtex4} and \cite{erlangen} or even bio-inspired hardware using the POE-model \cite{poe} as discussed in Section \ref{sec:related}.
More recent papers put effort in a combined approach by either implementing autonomic systems on reconfigurable architectures \cite{reconfigurable} or creating evolvable systems via self-aware applications \cite{evolvable}. In this section software based designs, hardware based designs are co-designs are discussed. 

%% -- Software based papers --------------------------
\subsection{Software based flexible approaches}
\label{sec:software}
%%Survey of Frameworks, Architectures and Techniques in Autonomic Computing

Self-configuration is the ability of the system to perform configurations according to the pre-defined high level policies and seamlessly adapt to change caused by automatic configurations. Self-optimization is the ability of the system to continuously monitor and control resources to improve performance and efficiency. Self-healing is the ability of the system to automatically detect, diagnose and repair faults. Self-protection is the ability of the system to pro-actively identify and protect itself from malicious attacks or cascading failures that are not corrected by self-healing measures.

This paper presents a wide variety of techniques in autonomic computing. Hot-swapping is a technique to inject monitoring and diagnostic code into live code using inter-positioning and replacement. Data clustering is an unsupervised learning algorithm, used to identify configuration classes and determines the degree of similarity between clusters using convex average metric.

				% April, 2009, maybe use this only for prior knowledge
% Software based approaches
%
\cite{selfaware} presents an implementation of a FPGA-based self-aware adaptive computing system which blends several software based techniques such as \emph{monitoring, decision making} and \emph{self-adaptation}. This system is built on top of a heterogeneous architecture enabling and utilizing the effectiveness of these systems. The operating system on top enables performance monitoring and can take actions involving software and hardware adaptation. The \emph{application Heartbeats} is used to asses and monitor performance and progress and the \emph{Implementation Switch Service} is used to switch between different implementations of the same functionality using a heuristic based decision mechanism. 

Based on constraints such as available resources and input data size and type the operating system chooses the best implementation at run-time. This decision is based on the \emph{observe-decide-act (ODA) loop}. Given certain performance goals these loops \emph{observe} the current state, \epmh{decide} upon an action and \emph{acts} accordingly as will be discussed in Section \ref{sec:selfawareness}, \ref{sec:decisions} and \ref{sec:selfadapting}.

The Heartbeats application is a monitoring application which makes it possible to assert performance goals as heart-rate windows delimited by a minimum and maximum performance, or \emph{heart-rate} \cite{evolvable}. The Heartbeats API is made of small straightforward functions and allows declaring performance goals. Software components first have to register, specifying parameters such as minimum and maxmim heart-reate, size of the windows of observation and history buffers \cite{selfaware}. The application then updates the progress of the execution calling the function that signifies a heartbeat \cite{evolvable}.

The implementation switch service presented in \cite{selfaware} inspired by the \emph{hot-swap mechanism} is fundamental for the system to switch between different implementations of the same functionality at run-time. Since threads can access data structures concurrently and data structures can differ for different implementations, \cite{selfaware} proposes a framework to account for these shortcomings in the hot-swap mechanism.  This framework is divided in three sub-phases: a prior phase representing the common work scenario, a transfer phase that blocks new requests in order to reach a quiescent state, translated to fit another data structure and a post phase in which the blocked and new requests are allowed to proceed. 
%
%[SMARTLOCKS PAPER] presents a combination of the Heartbeats application in cooperating with other frameworks. Smartlocks is a spinlock library that will adapt its internal implementations based on defined goals during run-time using heurstics as discussed in subsection \ref{sec:decisions} and machine learning. 
%
%\cite{reconfigurable} dscusses using adaptive programming in situations where input changes lead to relatively small output changes. They present a hardware-software codesign paradigm that develops a new performance model and associated evaluation metrics to differentiate between various levels of performance across different portions of software modules. Incoorporating this into a codesign environment increases the flexibility of the system. 
%distributed self trained algorithms, machine learning, heartbeats application
%adaptive programming
%
%% -- Hardware based papers --------------------------
\subsection{Hardware based fast approaches}
\label{sec:hardware}
%
%% A Phylogenetic, Ontogenetic, and Epigenetic View of Bio-Inspired Hardware Systems

Inspired by life on Earth and the natural processes of living organisms, \emph{bio-inspired} hardware systems have evolved. \cite{poe} introduces the POE model, that classifies bio-inspired systems according to three axes:
\begin{itemize}
	\item Phylogeny, which concerns the evolution of a species over time, aiming for optimization of the genome
	\item Ontogeny, which is about a second biological organization within multicellular organisms, being celluar reproduction
	\item Epigenesis, concerning the learning systems of organisms such as the nervous system, the immune system and the endocrine system.
\end{itemize}

Along the axis of phylogeny, evolvable hardware can be found. Artificial evolution and large-scale programmable circuits are the two underlying themes. Evolutionary algorithms are common nowadays, and strive for optimization and automatic programming. Programmable circuits are integrated circuits that are to be configured by the user. Back in the days, the PAL (Programmable Array Logic) was the most popular PLD (Programmable Logic Devices). Nowadays, FPGAs (Field Programmable Gate Arrays) have become widely used, providing high flexibility and the possibility of being reconfigurable.

Ontogenetic systems have the ability to self-repair as the main goal of ontogeny is growth, or construction. Whereas phylogeny is about reproduction of the system by use of crossover and mutation, ontogeny is about replication by creating daughter cells that are exact duplicates of the mother cell.

Epigenetic systems are more software-based, as they contain error detection and immune systems for computers. It becomes interesting when multiple axes are being combined, as stated at the end of \cite{poe}. By dreaming about POE hardware systems that are endowed with evolutionary, reproductive, regenerative learning capabilities this article is often consulted in research considering evolvable systems.					% August, 2002
% The Erlangen Slot Machine: A Dynamically Reconfigurable FPGA-based Computer

[There has to be some images added or else it doesn't come across, or just less info] 

The Erlangen Slot Machine from \cite{erlangen} tackles the four major limitations of the Virtex-II FPGA produced by Xilinx. First, the I/O dilemma caused by fixed pins spread around the device is solved by connecting all bottom pins from the FPGA to an interface controller realizing a crossbar. It connects FPGA pins to peripherals automatically based on the slot position of a placed module. This I/O rerouting principle is done without reconfiguration of the crossbar FPGA.

Second, the memory dilemma has been solved. In normal Virtex-II FPGAs, a module can only occupy the memory inside its physical slot boundary. Storing data in off-chip memories is therefore the only solution. In the ESM, six SRAM banks are connected to the FPGA. Since these banks are placed at the opposite side as the crossbar, a module will connect to peripherals from one side, while the other side will be used for temporally storing computational data. In order to use a SRAM bank (called a slot), the module must have at least a width of three micro-slots, in which the total device is divided. This organization simplifies relocation, enabling a partially reconfigurable computing system. Also, equal resources will be available for each module.

Finally, the inter-module communication dilemma is dealt with. Dynamically routing signal lines on the hardware is a very difficult task. The ESM uses a combination of bus-macros, shared memory, RMB (Reconfigurable Multiple Bus) and a crossbar to take away the limiting factor for the wide use of partial dynamic reconfiguration.

In order to initialize executable application modules and their run-time supervision the ESM requires an operating system. This is being implemented by means of a Reconfiguration Manager that uses a parallel port interface to download and store bitstreams into the flash memory. A pipelined data flow architecture has been used, replacing the finite state machine by a MicroBlaze microcontroller and employing a data crossbar between plug-ins. Providing  a new architecture to avoid the current physical problems of reconfigurable FPGAs, a new inter-module communication concept, as well as an intelligent module reconfiguration management has made the ESM an alternative for the Xilinx FPGAs.
			% April, 2007
% A Direct Bitstream Manipulation Approach for Virtex4-based Evolvable Systems

In \cite{virtex4} and \cite{dpr} a Virtex-4 FPGA implementation is introduced for evolvable systems. Other then earlier versions of Virtex (e.g. the Virtex-II Pro), Virtex4 devices enable two-dimensional dynamic reconfiguration, a feature which considerably reduces the reconfiguration time and thus the evolution time (\cite{virtex4}). A big limitations of Virtex FPGAs is an almost unknown and undocumented bitstream data format and an unsafe configurations schema. By using both VRCs (Virtual Reconfigurable Circuits) and direct bitstream manipulation, this architecture eliminates this limitation. This Virtex-4 based device, which takes advantage of 2D reconfiguration capabilities and direct manipulation of the bitstreams is the first one of its kind and will be discussed in Section \ref{sec:candidate}.				% May, 2010
%
%% -- Combinations based papers ----------------------
\subsection{Co-design based systems}
\label{sec:codesign}
%
% A Fast Reconfigurable 2D HW Core Archtiecture on FPGAs for Evolvable Self-Adaptive Systms

Another Virtex-4 based architecture introduced in \cite{dpr} also applies the 2D reconfiguration core. Rather than direct bitstream manipulation, a \emph{dynamically partial reconfiguration (DPR)} control engine has been integrated. As a result, the processing elements (PEs) of the reconfigurable core are structured as a 2D systolic array, known for its high performance and restrained use of resources. The reconfiguration engine has been optimized by applying three enhancements that will be discussed in \ref{sec:dpr}.					% June, 2011
\\
\\
% From reconfigurable architectures to self-adaptive autonomic systems ++

During the design space exploration phase of system design, overheads associated with reconfiguration and hardware/software interfaces need to evaluated carefully to harvest full potential. The context where different applications demand ever increasing adaptability and performance has already be answered by introducing reconfigurable SoC employing different multiprocessor cores. The increasing prominence of reconfigurable devices within such systems require hardware/software co-design for SoC to address the trade-off between software execution and reconfigurable hardware.

Right now this co-design emphasizes on identifying intensive kernel tasks and implementing these tasks on the reconfigurable hardware. The performance model of the hardware depends on the degree of parallelism while the performance model for software execution is static and does not become affected by external factors.
There are several modes and interfaces to configure a specific FPGA family: the JTAG cable, the selectMAP interface for daisy chaining the configuration on multiple FPGAs, configuration loading from PROMs or compact ash cards, micro controller based configuration and internal configuration access port (ICAP).

Partial dynamic reconfiguration is a key feature that makes FPGAs unique. This addresses the lack or resources to implement an application and its adaptability needs. It could also be achieved by having a larger resource array, but this is not always viable for non-trivial designs. Reconfigurable hardware taking advantage of partial dynamic reconfiguration is the perfect trade-off between the speed of HW and the flexibility of SW. Other aspects that motivate the use of online self-adaptable systems are the QoS and the reliability and continuity of the service.

Important problems are the lack of availability of software tool chains that take into account partial dynamic reconfiguration, the time overhead the reconfiguration process introduces and the two-dimensional partitioning strategy reconfigurable devices need: spatial and temporal.

Scenario's where self-awareness is useful include: mobile technologies, cloud-computing systems, adaptive and dynamic compilation, multi-core micro-architectures and novel operating systems. Existing adaptive systems often fail because of they are largely ad hoc and fail to incorporate true goals.

Since the introduction of reconfigurable hardware platforms such as FPGAs, the hardware domain shifted into the software domain: the possibility of implementing a reconfigurable architecture increased the flexibility of the hardware.

In online task management the operating system requests for a hardware module by configuring the IP core on the FPGA and creating a communication channel between the module and the software application in a transparent way. This process is managed by two kernel extensions called the reconfiguration support and the centralized reconfiguration manager. This centralized approach allows for module caching or module allocation and to enhance parallel execution of hardware IP-cores.

The results of the system however shows the negative impact of reconfiguration latency in the execution of a functionality, therefor there is a big need to evaluate this overhead carefully, it is not always an improvement. This is basically the result of the paper.

\\
% Evolvable systems on reconfigurable architecture via self-aware adaptive applications. ++

\cite{evolvable} proposes an evolvable system exploiting self-adaptive techniques by running a customized version of GNU/Linux and a set of adaptive applications on top of a heterogeneous system featuring a multi-core Intel Core i7 processor and a reconfigurable devide, a Xilinx Virtex 5 FPGA. The online reconfiguration mechanism they propose is inspired by the modern object oriented operating system called K42, developed by IBM. The underlying hardware architecture is made up of static area, containing the general purpose processor and a dynamic area which contains the FPGA. 

The customized operating system provides common functionalities through standard libraries, but is also able to choose at runtime the best implementations for the required functionalities thanks to a set of self-adaptive libraries. 

transfers a switchable unie in a quiescent state, transferres the state and changes the references. 

The switchable units in the hot-swap method in \cite{evolvable} are identified as the libraries that export an implementation of a certain functionality. The self-adaptive library or Dynamic-link library(DLL) and the software implementation library target the reconfigurable FPGA and multicore respectively. 

